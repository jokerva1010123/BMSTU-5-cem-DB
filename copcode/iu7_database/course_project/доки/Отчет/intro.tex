\newpage
\section*{Введение}
\addcontentsline{toc}{section}{Введение}

Информационные технологии стали неотъемлемой частью жизни современного человека. 
Сегодняшние читатели могут получить знания в разных направлениях, в том числе  за счет нетрадиционных форм доступа к информационным ресурсам. Но как быть с уже существующими книгами? 
У большинства людей дома есть домашняя библиотека, в которой находится разное количество книг, и не всегда можно с легкостью запомнить их расположение.
\hfill \break

  Целью данной курсовой работы является создание клиент–серверного приложения <<Домашняя библиотека>>, которое предоставляет доступ к книгам домашней библиотеки и возможность просмотра ее местоположения.
  
  Актуальность разработки состоит в том, что с помощью такого приложения значительно уменьшается трудоемкость ведения учета информации о книгах и их поиск.
  
  Для выполнения цели необходимо выполнить следующие задачи:
  \begin{itemize}
	\item формализовать задачу в виде определения необходимого функционала;
	\item провести анализ существующих СУБД;
	\item спроектировать базу данных, необходимую для хранения и структурирования данных;
	\item реализовать спроектированную базу данных с использованием выбранной СУБД;
	\item реализовать приложение для взаимодействия с реализованной БД.
\end{itemize}

