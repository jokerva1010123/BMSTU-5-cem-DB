\chapter{Группировка (Лекция 4)}

SP GROUP (Pno, Qny) as PQ

\section{Реляционное сравнение}

<рел. выр> <опер. сравн.> <рел выр.> опер. сравн. >= супермножество
> - срьств. супермножество

<расширение> - добаление новых атрибутов по горизонтали.

<расширение> ::= EXTEND <реляц. выр.> ADD <список добавляемых расширений>

представляет собой выражение, после которого следует ключевое слово AS (Проименованное выражение)

<добавл расшир> ::= <выражение> AS <имя атрибута>

Обобщение - горизонтально группирует записи.

<Обобщение> :== SUMMARIZE <реляц выр> PER <рел выр> ADD <список добавляемых обобщений>

	<добавл обобщение> :== <тип обобщения> [(скалярн выражение) AS <имя атрибута>]

<тип обобщения> :== COUNT | MIN | MAX | ANG | SUM | ALL | ANY

\section{Исчисление кортежей}

<объявление кортежной переменной> :== RANGE OF <переменная> IS <список областей>

<область> :== <отножение> | <рел выражение>

<рел выр> :== (список целевых элементов) [WHERE wff]

<целевой элемент> ::= переменная | переменная атрибут [AS <псевдоним>]

правильно построенная  функция wff ::= условие | NOT wff | условие AND wff | if условие then wff |
exists переменная (wff) | FORAll переменная (wff) | (wff)


\chapter{Семинар 3}

\section{Объекты БД}

\begin{itemize}
	\item table (+ temp)
	\item view
	\item constraints
	      \begin{itemize}
		      \item PK/FK
		      \item default, not null, cheek
	      \end{itemize}
	\item function
\end{itemize}

Функции.

\begin{enumerate}
	\item По поведению:
	      \begin{enumerate}
		      \item детерминированные\\
		            (Пример:\\
		            select add date('day', '2020-10-03'))
		      \item нет (Пример:\\
		            getDate())
	      \end{enumerate}
	\item По возвращаемому значению
	      \begin{enumerate}
		      \item Скалярная функция (\textbf{Синтаксис}:\\
		            create function [схема].имя (<параметры>)\\
		            returns <ск.тип>\\
		            $[with <$опции$>]$\\
		            $[$as$]$\\
		            begin\\
		            <тело>\\
		            return <ск. переменная>
		            end $[$;$]$)\\
		            (\textbf{Пример:}\\
		            create functions dbo.AvgPrice()\\
		            returns smallmoney\\
		            with schemabinding\\
		            as\\
		            begin\\
		            return (select avg(Cnt) from P)\\
		            end;\\)
		            (\textbf{Пример:}\\
		            create function dbo.PriceDiff(@Pricesmallmoney)\\
		            returns smallmoney\\
		            as\\
		            begin\\
		            return @Price\_dbo AvgPrice()\\
		            end)\\
		            select Price.dbo.AfgPrice()\\
		            dbo.PriceDiff(Price)\\
		            from P
		      \item Подставляемая функция.\\
		            \textbf{Синтаксис:}
		            create function $[$схема$]$.имя (<парам>)\\
		            returns table\\
		            $[$with <опции>$]$\\
		            $[$as$]$\\
		            return <sql-запрос>\\
		            end $[$;$]$\\
		            \textbf{Пример:}
		            create function dbo.fullSpy()\\
		            return table\\
		            as\\
		            return (select Sname, Pname\\
		            from S join SP on S.Sno = SP.Pno\\
		            join P on SP.Pno=P.Pno)\\
		            where P.color=@color
		      \item  Многооператорная функция.\\
		            \textbf{Синтаксис:}
		            create function $[$схема$]$.имя (<парам>)\\
		            returns @возвр.перем. table <опред. таблицы>\\
		            $[$with <опции>$]$\\
		            $[$as$]$ declare\\
		            begin\\
		            <тело>\\
		            return\\
		            end$[$;$]$\\
		            \textbf{Пример:}\\
		            create function dbo.FnGetReport(@id as id)\\
		            returns @Reports table (eid int, repid int)\\
		            as\\
		            begin\\
		            declare @Empid as int\\
		            select @Empid = 1\\
		            select id from \dots\\
		            into @Empid\\
		            insert into @Reports(\dots) values(\dots)
	      \end{enumerate}
\end{enumerate}