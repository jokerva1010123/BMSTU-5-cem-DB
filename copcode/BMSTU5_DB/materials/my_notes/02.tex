% Глава.	
\chapter{Реляционная модель (Лекция 2)}

\section{ER - модель}

\begin{itemize}
	\item Сущность
	\item Связь
\end{itemize}

Объекты обозначаются прямоугольниками. Внутри пишем название.

\textbf{Виды сущностей:}
\begin{itemize}
	\item Сильные - Обозначаются просто в рамке.
	\item Слабые - не могут существовать друг без друга. Факультет и предметы.
	      Обозначается вложенным квадратом (рамочка).
\end{itemize}

\textbf{Атрибуты} отображаются овалами. Внутри пишем название атрибута.

\textbf{Виды связей:}
\begin{itemize}
	\item Один к одному. Студент-зачетка.
	\item Один ко многим. Статья-рецензия. Добавляем внешний ключ со стороны многих.
	      Из многих в сторону одного.
	\item Многие ко многим. Студент-преподаватель. Добавляем связочную таблицу.
\end{itemize}

\section{Реляционная модель}

\textbf{Реляционная модель}
\begin{itemize}
	\item Структурная часть - отвечает за то, какие объекты есть.
	\item Целостная - отвечает за ссылки. DDL.
	      \begin{itemize}
		      \item Ссылочная целостность (FK)
		      \item Целостность сущности (PK) - говорит о том, что есть первичный ключ.
		            Нет повторения. Всегда знаем на что ссылаемся.
	      \end{itemize}
	\item Манипуляционная - за механизм работы с данными. DML.
\end{itemize}

\textbf{Домен} = (примерно равно) тип данных.\\
\textbf{Атрибут} (отношения) = (примерно равно) столбец. Упорядоченная пара вида:\\
имя-атрибута,имя-домена\\
\textbf{Схема отношений} = (примерно равно) Заголовок. имя-отношение, имя-домена\\
\textbf{Кортеж} = (примерно равно) Строка. Имя-атрибута, значение-атрибута
\textbf{Отношение} = (примерно равно) таблица.

Непустое подмножество множества атрибутов схемы отношения будет \textbf{потенциальным ключом} тогда и только тогда,
когда оно будет обладать свойствами:
\begin{itemize}
	\item уникальности (в отношении нет двух различных кортежей с одинаковыми
	      значениями потенциального ключа)
	\item неизбыточности (никакое из собственных подмножеств множества
	      потенциального ключа не обладает свойством уникальности
\end{itemize}

\textbf{Внешний ключ} в отношении R2 – это непустое подмножество множества атрибутов FK этого отношения, такое, что:
\begin{itemize}
	\item Существует отношение R1 (причем отношения R1 и R2 необязательно различны) с потенциальным ключом CK;
	\item Каждое значение внешнего ключа FK в текущем значении отношения R2 обязательно совпадает со значением ключа CK
	      некоторого кортежа в текущем значении отношения R1.
\end{itemize}


\chapter{Реляционная алгебра (Лекция 3)}

Реляционная алгебра - замкнутая система.

Реляционная алгебра является основным компонентом реляционной модели, опубликованной Коддом, и состоит из
восьми операторов, составляющих две группы по четыре оператора:

\begin{itemize}
	\item Традиционные
	      \begin{itemize}
		      \item Объединение. (union)
		      \item Пересечение. (intersect)
		      \item Вычитание. (minus) (В mysql иначе называется)
		      \item Декартово произведение - все со всеми.  (times)
	      \end{itemize}
	\item Специальные
	      \begin{itemize}
		      \item Проекция. (PROJECT, []) помогает выбирать не все из нашего отношения.
		            Можно набрать только те атрибуты, которые будем использовать далее.
		      \item Фильтрация. (WHERE)
		      \item Соединения. (JOIN)
		      \item Деление. (DIVIDE BY)
	      \end{itemize}
\end{itemize}

% \begin{table}[ht]
% 	\centering
% 	\begin{tabular}{ | l | l |}
% 		\hline
% 		id & name   \\ \hline
% 		1  & Иван   \\ \hline
% 		2  & Петр   \\ \hline
% 		3  & Степан \\ \hline
% 		\hline
% 	\end{tabular}
% 	\caption{Таблица тестов}
% 	\label{table:ref1}
% \end{table}

% \begin{lstlisting}[label=some-code,caption=Код подпрограммы sub\_1]
% \end{lstlisting}

\textbf{Деление. (DIVIDE BY)}

R1 \{A, B\}

R2 \{B\}

R1 DIVIDE BY R2 = R1[A] minus(R2 YIMER R1[A]) minus R1)[A]

\section{Синтаксис реляционной алгебры}

Любое реляционное выражение - это унарное выражение или бинарное выражение

\begin{itemize}
	\item Унарное - выражение с одним элементом.
	      \begin{itemize}
		      \item Переименование := терм RENAME  имя\_атрибута AS новое\_имя\_атрибута
		      \item Ограничение := терм WHERE логическое\_выражение
		      \item Проекция := терм | терм[список атрибутов]
	      \end{itemize}
	\item Бинарное - с двумя элементами
	      \begin{itemize}
		      \item Бинарное выражение := проекция бинарная\_операция (реляционное\_выражение)
		      \item Бинарный операция := UNION | INТERSECT | MINUS | TIМES | JOIN | DIVIDEBY
	      \end{itemize}
\end{itemize}

Терм - либо отношение, либо другое реляционное выражение.
Реляционное выражение всегда берется в круглые скобки.

Имеются таблицы:
\begin{itemize}
	\item S - поставщик
	      S(Sno: integer, Sname: string, Status: integer, City: string)
	\item P - поставщик
	      P(Pno: integer, Pname: string, Color: string, Weight: real, City: string)
	\item SP - Таблица связка
	      SP(Sno: integer, Pno: integer, Qty: integer)
\end{itemize}

\begin{table}[ht]
	\centering
	\begin{tabular}{ | l | l | l | l |l |}
		\hline
		id & Имя детали & цвет & вес  & Город    \\ \hline
		1  & Гвоздь     & К    & 10.3 & Москва   \\ \hline
		2  & Винт       & З    & 15.8 & Рязань   \\ \hline
		3  & Гвоздь     & С    & 3.4  & Смоленск \\ \hline
		\hline
	\end{tabular}
	\caption{Детали}
	\label{table:ref1}
\end{table}


\begin{table}[ht]
	\centering
	\begin{tabular}{ | l | l | l | l |l |}
		\hline
		id & Имя поставщика & статус & город       \\ \hline
		1  & ООО Ромашка    & 5      & Рязань      \\ \hline
		2  & ООО Рубин      & 3      & Красногорск \\ \hline
		\hline
	\end{tabular}
	\caption{Поставщики}
	\label{table:ref1}
\end{table}

\section{Примеры}

Реляционные алгебра. Выражения.

1. Получить имена поставщиков, которые поставляют деталь под номером 2.

\begin{lstlisting}[label=some-code,caption=Пример 1]
	(( S JOIN SP ) WНEPE Рno = 2 ) [ Sname ]
\end{lstlisting}

\begin{lstlisting}[label=some-code,caption=Пример 1]
	select Sname
	from S
	join SP on S.Sno = SP.Sno
	where SD.Pno = 2
\end{lstlisting}

Пример 1.2 быстрее

\begin{lstlisting}[label=some-code,caption=Пример 1.2]
	((SP where Pno=2) join S) [Sname]
\end{lstlisting}


2. Получить имена поставщиков, которые поставляют по крайней мере одну красную деталь.

\begin{lstlisting}[label=some-code,caption=Пример 2]
	((( Р WНERE Color = 'Красный' ) JOIN SP) 
	[ Sno ] JOIN S ) [ Sname ]
\end{lstlisting}


\begin{lstlisting}[label=some-code,caption=Пример 2]
	select Sname
	from S
	join SP on S.Sno = SP.Sno
	join P on P.Pno = SP.Pno
	where color='К'
\end{lstlisting}


3. Получить имена поставщиков, которые поставляют все детали.

\begin{lstlisting}[label=some-code,caption=Пример 3]
	(( SP [ Sno, Рno] DIVIDE BY Р 
	[ Рno ] JOIN S ) [ Sname ]
\end{lstlisting}

\begin{lstlisting}[label=some-code,caption=Пример 3]
	with group SP(Sno, cnt) as (
		select Sno, count(distinct Pno)
		from SP
		group by Sno
	)
	select Sname
	from group SP (тут не обязательно as) gSP 
	join S on gSP.Sno=S.Sno
	where cnt=(select count(distinct Pno)
	from P)	
\end{lstlisting}

4. Получить номера поставщиков, поставляющих по крайней мере все те детали, которые поставляет поставщик под
номером 2.

\begin{lstlisting}[label=some-code,caption=Пример 4]
	SP [Sno, Рno] DIVIDE ВY 
	(SP WНEPE Sno = 2)[ Рno ]
\end{lstlisting}

\begin{lstlisting}[label=some-code,caption=Пример 4]
	with group SP(Sno, cnt) as (
		select Sno, count(distinct Pno)
		from SP
		where Sno in ()
		group by Sno in (select count(distinct Pno)
		from SP
		where Sno=2)
	)
	select Sname
	from group SP (тут не обязательно as) gSP 
	join S on gSP.Sno=S.Sno
	where cnt=(select count(distinct Pno)
	from SP
	where Sno=2)	
\end{lstlisting}

5. Получить все пары номеров поставщиков, размещенных в одном городе

\begin{lstlisting}[label=some-code,caption=Пример 5]
	((( S RENAМE Sno AS FirstSno ) 
	[ FirstSno, City ] JOIN
	(S RENAМE Sno AS SecondSno ) 
	[ SecondSno , City ])
	WНEPE FirstSno < SecondSno )
	[ FirstSno, SecondSno ]
\end{lstlisting}

\begin{lstlisting}[label=some-code,caption=Пример 5]
	select firstS.Sno, SecondS.Sno
	from S firstS inner join S secondS
	on firstS.Sno = secondS.Sno
	where firstS.Sno < SecondS.Sno 
	(Эта фильтрация избавит от дублей)
\end{lstlisting}


6. Получить имена поставщиков, которые не поставляют деталь под номером 2.

\begin{lstlisting}[label=some-code,caption=Пример 6]
	((S[ Sno ] MINUS (SP WНEPE Рno = 2 )
	 [ Sno ] ) JOIN S ) [Sname]
\end{lstlisting}

\begin{lstlisting}[label=some-code,caption=Пример 5]
	select Snp from S
	minus
	select distinct Sno
	from SR
	where Pno=2
\end{lstlisting}


\chapter{Семинар 2}

Таблица P.

\begin{table}[ht]
	\centering
	\begin{tabular}{ | l | l | l | l |l |}
		\hline
		id & Pname  & Color & Weight & City     \\ \hline
		1  & Гвоздь & К     & 10.3   & Москва   \\ \hline
		2  & Винт   & З     & 15.8   & Рязань   \\ \hline
		3  & Гвоздь & С     & 3.4    & Смоленск \\ \hline
		4  & шуруп  & К     & 11     & Рязань   \\ \hline
		5  & шайба  & с     & 17.8   & Смоленск \\ \hline
		\hline
	\end{tabular}
	\caption{Таблица деталей P.}
\end{table}

Таблица поставщика - S

\begin{table}[ht]
	\centering
	\begin{tabular}{ | l | l | l | l |l |}
		\hline
		Sno & Sname                  & Status & City     \\ \hline
		1   & ООО Ромашка            & 5      & Москва   \\ \hline
		2   & ООО Рубин              & 3      & Рязань   \\ \hline
		2   & ООО Зеленоглазое такси & 4      & Смоленск \\ \hline
		\hline
	\end{tabular}
	\caption{Таблица поставщика - S}
	\label{table:ref1}
\end{table}


\begin{table}[ht]
	\centering
	\begin{tabular}{ | l | l | l |}
		\hline
		Sno & Pno & Cnt \\ \hline
		1   & 1   & 100 \\ \hline
		2   & 1   & 150 \\ \hline
		3   & 1   & 180 \\ \hline
		1   & 2   & 180 \\ \hline
		3   & 2   & 180 \\ \hline
		4   & 3   & 180 \\ \hline
		5   & 3   & 180 \\ \hline
		\hline
	\end{tabular}
	\caption{Таблица SP}
	\label{table:ref1}
\end{table}


\textit{Агрегатная функция} - sum, max, min, count,

\begin{lstlisting}[label=some-code,caption=Пример]
	select Color, count(*)
	from p
	group by Color
\end{lstlisting}


having используется для фильтрации групп.

\begin{lstlisting}[label=some-code,caption=Пример]
	select Color, count(*)
	from p
	group by Color
	having count(*)>1
\end{lstlisting}

order by - сортировка. Есть прямой порядок () и обратный (desc).
По умолчанию по возрастанию.

Отсортируем таблицу деталей по весу.

\begin{lstlisting}[label=some-code,caption=Пример]
	select Color, count(*)
	from p
	order by Pname, Weight desc;
\end{lstlisting}

Порядок записи инструкций.
Цифрами показан порядок выполнения.


\begin{lstlisting}[label=some-code,caption=Порядок записи инструкций]
select (5)
from (1)
where (2)
group by (3)
having (4)
order by (6)
\end{lstlisting}

Нерабочий пример (потому что имя задаем на этапе позже)
На этапе where использовать псевдонимы, которые мы создаем в select нельзя.

(2) - выполнится вторым действием,но у нас еще нет псевдонима. (Пример 6.5)

\begin{lstlisting}[label=some-code,caption=Пример]
select Pname as myName
from P (1)
from where myName ='Гвоздь' (2)
order by myName
\end{lstlisting}

\begin{lstlisting}[label=some-code,caption=Пример внутренних запросов]
select Sname 
from S
where Sno in 
(select distinct Sno
from SP
where Pno=2)
\end{lstlisting}

Запрос - найти цвет с max кол-вом деталей.
Действия
\begin{itemize}
	\item Сначала группируем по цвету. grop by Color
	\item Найти max.
	\item Вернуться к табл. и найти
\end{itemize}

with - показывает, что след. запрос будет выполняться до select.
Это только для запроса.

\begin{lstlisting}[label=some-code,caption=with. найти цвет с max кол-вом деталей.Обобщенное табличное выражение]
with group Color(Color, cnt) as
select color count(*)
from P
group by Color(
(select max(cnt)
from (select Color,  count(*) as cnt
from P
grop by Color))
)
select Color 
from group Color 
where cnt in (
select max(cnt)
from group Color
)
\end{lstlisting}

Теперь переходим на \textit{JOIN} - соединения

Виды:
\begin{itemize}
	\item Внутренний. inner join (Это пересечение на кругах Эллера.);
	\item Внешние. outer join. (3 вида)
	      \begin{itemize}
		      \item left join (На кругах это весь круг A)
		      \item right join (На кругах это весь круг B)
		      \item full join(полное) (На кругах это оба круга (и A и B))
	      \end{itemize}
\end{itemize}

\begin{lstlisting}[label=some-code,caption=Внутреннее соединение]
select A.id, B.id, A.name, B.fio
from A join B
on A.id = B.id
\end{lstlisting}

Виды join, которыми мы сможем воспользоваться
\begin{itemize}
	\item Nested loops join. (Сложность n*n).
	      Можно нашей СУБД указать, что нужно использовать её.
	      Минусы: избыточность.
	\item Hash join (Сложность n+n). Можно сравнивать только на равенство.
	      Минусы: доп расходы на таблицу.
	\item Merge join (Изначально таблицы должны отсортированы по ключу.)
	      Минусы: нужно сортировать
\end{itemize}

Операция над множествами. Тело - это множество кортежей.

Ниже представленный запрос даст таблицу с двумя столбцами (id, name).
Атрибуты называются по верхней схеме.

\begin{lstlisting}[label=some-code,caption=union]
select id, name, from A
union [all]
select id, FIO from B;
\end{lstlisting}

\begin{lstlisting}[label=some-code,caption=minus]
select id, name, from A
minus
select id, FIO from B;
\end{lstlisting}

join - добавляет столбцы, union - дописывает в конец.